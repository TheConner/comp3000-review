\section{How2Shell}

How a shell works (generally):
\begin{itemize}
\item A shell is just another program.
\item What happens when a user runs a program:
  \begin{itemize}
  \item Based on the \texttt{PATH} environment variable, the shell will search for
    binaries that correspond to the user's requrest.
  \item If the user specified output redirection, the shell will redirect STDOUT
    / STDIN appropriately (see section on this)
  \item Fork the current process, the child proccess will execute the specified
    program.
  \item The child process sends a signal to the parent process (shell), shell
    continues execution.
  \end{itemize}
\end{itemize}
There are exceptions for I/O redirection, piping, and running programs in the
background, however the above workflow applies to all of those cases.

\section{Working Dir}
%% * cd, working directory

The working directory represents where your program, or shell, currently is.
\begin{itemize}
\item To change the current working directory in bash, use the \texttt{cd}
  command.
\item To change the current working directory in C, use the \texttt{chdir()} function.
\end{itemize}

\section{Globbing and Wildcards}
%% * globbing, who does it

Globbing and wildcard matching is something done by the shell, with logic
  implemented by the C library \texttt{glob.h}, or the logic can be defined
  manually. It dosen't really matter as long as it behaves like a normal glob

\subsection{How To Glob}

Consider the following files:

\begin{verbatim}
book1 book2 book3 potato1 potato2 potato3
\end{verbatim}

To get all books, a few possible wildcard statements would be \texttt{b*},
\texttt{book*}, etc. If I wanted to get all things with a 3 I could do
\texttt{*3}.
If I wanted to get everything containing the letter o and the number 2, I could
do \texttt{*o*2}

Globbing isn't understood by a program, so all of these glob statements
\textbf{expand} out to be a list of files, i.e
\begin{verbatim}
b* = book1 book2 book3
\end{verbatim}

