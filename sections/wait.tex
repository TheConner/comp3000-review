
\section{wait()}
%% * wait
%%   - when it waits
%%   - when it doesn't
%%   - what it really is for

%   man 2 wait
%       All  of  these  system  calls  are used to wait for state changes in a child of the calling process, and obtain information about the child whose state has changed.  A
%       state change is considered to be: the child terminated; the child was stopped by a signal; or the child was resumed by a signal.  In the case of  a  terminated  child,
%       performing  a  wait  allows  the system to release the resources associated with the child; if a wait is not performed, then the terminated child remains in a "zombie"
%       state (see NOTES below).
%
%       If a child has already changed state, then these calls return immediately.  Otherwise, they block until either a child changes state or a signal handler interrupts the
%       call  (assuming  that  system calls are not automatically restarted using the SA_RESTART flag of sigaction(2)).  In the remainder of this page, a child whose state has
%       changed and which has not yet been waited upon by one of these system calls is termed waitable.

Wait is used to wait for a change of state in a child process, and to obtain
information about the child process whose state has changed.

A state change is considered to be:
\begin{itemize}
\item The child terminated
  \begin{itemize}
  \item In this case, performing a wait allows the system to release the
    resources associated with the child who terinated. \textbf{If a wait is not
      performed, the child is considered to be a ``zombie''}
  \end{itemize}
\item The child was stopped by a signal
\item The child was resumed by a signal
\end{itemize}

If the child has already changed states then a call to wait will return
immediately. Else they will block until either a child changes state or a signal
handler interrupts the call.

