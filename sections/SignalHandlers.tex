
\section{Signal Handlers}
%% * signal handlers
%%   - how to define
%%   - when called
%%   - interrupting system calls

\subsection{How To Define and Register}

Typically a signal handler is defined in a method, and looks like this:
\begin{csrc}
void signal_handler(int the_signal)
{
  int pid, status;

  if (the_signal == SIGHUP) {
    fprintf(stderr, "Received SIGHUP.\n");
    return;
  }
        
  if (the_signal != SIGCHLD) {
    fprintf(stderr, "Child handler called for signal %d?!\n",
            the_signal);
    return;
  }

  pid = wait(&status);

  if (pid == -1) {
    /* nothing to wait for */
    return;
  }

  if (WIFEXITED(status)) {
    fprintf(stderr, "\nProcess %d exited with status %d.\n",
            pid, WEXITSTATUS(status));
  } else {
    fprintf(stderr, "\nProcess %d aborted.\n", pid);
  }
}
\end{csrc}
In the above example we have a generic method that takes an int (since that's
all a signal is) and then we provide logic depending on the signals we want to handle.

Signal handlers are typically registered when a program starts up, in the main
method.
\begin{csrc}
int main(int argc, char *argv[], char *envp[])
{
  // A struct for signal handlers
  struct sigaction signal_handler_struct;

  // Fill the struct with 0s
  memset (&signal_handler_struct, 0, sizeof(signal_handler_struct));

  // Assuming we have a signal_handler method
  signal_handler_struct.sa_handler = signal_handler;

  // Provide  behavior  compatible with BSD signal semantics by making certain system calls restartable across signals.
  signal_handler_struct.sa_flags = SA_RESTART;

  // Now we can register our signals to our signal handler.
  if (sigaction(SIGCHLD, &signal_handler_struct, NULL)) {
    fprintf(stderr, "Couldn't register SIGCHLD handler.\n");
  }

  if (sigaction(SIGHUP, &signal_handler_struct, NULL)) {
    fprintf(stderr, "Couldn't register SIGHUP handler.\n");
  }

  // ... rest of your program
}
\end{csrc}


