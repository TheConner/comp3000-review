\section{\texttt{mmap}}

\textbf{What is this for??}

From what we've used it for, file-backed mapping. If I had a file of size $n$
and I wanted to read it, my general algorithm would be allocate a buffer to read
the file, read file, play with file, life is great. However, if my file is size
$n$ and my memory is size $k$ where $n>k$, we have a problem.

mmap is handy because we never have to malloc anything - mmap will ``map'' the
file into memory without explicily reading it. Now, when you play with that
memory area \texttt{mmaped\_file[i]}, those memory operations will
\textbf{transparently} act on the file. It lets you map files into memory that
are larger than the available memory on the system using the OS's available
paging mechanism.

%% 
%% Couple of key variations
%%  - read-only mmap of file, normaly with PROT_EXEC
%%    * load code (executable, libraries)
%%    * read-only data (no PROT_EXEC)
%%    
%%  - read-write mmap, shared, of file
%%    * modifying file on disk
%%    
%%  - read-write mmap, private, of file
%%    * copy file to memory, modify memory
%%    
%%  - read-write mmap, anonymous, shared
%%    * allocate memory shared between processes
%% 
%%  - read-write mmap, anonymous, private
%%    * allocate memory private to process

There are a few ways of using MMAP:
\begin{itemize}
\item \texttt{PROT\_EXEC} -- used for loading executable code
\item R/W mmap, shared -- used for writing to and modifying files on disk
\item R/W mmap, private -- used for copying the file into memory and modifying
  the memory
\item R/W mmap, anonymous + shared -- used for allocating memory that's shared
  between processes
\item R/W mmap, anonymous + privte -- allocate memory to this process.
\end{itemize}


\section{Process Memory Map}
%% * where local vars, funciton arguments, global vars, stack, heap,
%% environment/command line args are all in a process's memory relative to each
%% other

From the lowest to the highest address:
\begin{itemize}
\item \textbf{Text segment}: contains executable instructions, this is placed
  below the heap or stack in order to prevent heaps and stack overflows from
  overwriting it.
\item \textbf{Initialized data}: above the text segment, contains global
  variables and static 
\item \textbf{Unitialized data / BSS}: BSS = block started by symbol. Data in
  this segment is initialized by the kernel before the program starts executing.
\item \textbf{Heap}: begins after the BSS segment, grows upwards towards the
  stack. Data in the heap is managed by malloc et al.
\item \textbf{Stack}: Sits at a high address, grows down towards the heap. the
  stack holds automatic variables, and \textit{stack frames} -- data about
  function calls and contexts. In general, we want to use the heap more than the
  stack because the stack ideally is used for execution-level stuff (frames)
\end{itemize}


