
\section{Umask}
%% * umask
%%   - how it affects open calls, which kinds (when you create a file)

Umasks, or file mode creation masks, are groupings of bits that define
permissions for files.

First, consider the model that is used for file permissions:
\begin{itemize}
\item User -- permissions for the immediate user
\item Group -- permissions for users that are in a files group
\item Others -- permissions for all other users 
\end{itemize}
For each class there corresponds a set of bits that represent the actions that
can be applied to the file:
\begin{itemize}
\item r -- read access to the file.
\item w -- write access to the file.
\item x -- permission to execute the file.
\end{itemize}

These permissions values can be represented as three bits, in the order RWX. For
instance to specify read, write, and not execute, the corresponding value will
be \texttt{110}.

\textbf{Translation to octal:} each set of these bits is effectively an octal
value (base 7). So all of these binary values can be represented as octal, for
example
\begin{itemize}
\item RWX-RWX-RWX = 111 111 111 = 777 permissions
\item RW--RW--R-- = 110 110 100 = 664 permissions
\end{itemize}

