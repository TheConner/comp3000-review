
\section{Environment Variables}
%% * environment variables
%%   - how they work
%%   - where they are in memory

Environment variables are constructed in the shell on login. These variables are
stored by the shell, and typically provide user-specific context and values to
processes.

Some examples of environment variables are:
\begin{itemize}
\item \texttt{PATH} contains the ``path'' to different areas on the system where
  binaries are located. A sample path could look like
  \texttt{/usr/local/bin:/usr/bin:/bin}, each of these three directories are
  searched by the shell to find programs
\item \texttt{USER} contains the username of the current user that's logged
    in.
  \item 
    \texttt{SHELL} contains the shell that the user is using
\item etc
\end{itemize}

You can set environment variables with \texttt{export} on bash, and printing
them by appending the variable name with a \texttt{\$}. For example, to set and
print a variable called \texttt{POOP}

\begin{verbatim}
export POOP=foo
echo $POOP
\end{verbatim}

